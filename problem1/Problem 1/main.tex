\documentclass{article}
\usepackage[utf8]{inputenc}

\title{Problem 1}
\author{Maziar Kosarifar}
\date{August 2019}

\begin{document}

\maketitle

\section*{Problem 1: Use linear/quadratic functions for verification}
Consider the ODE problem
$$u" + \omega^2 u = f(t), u(0) = I, u'(0) = V, t \in (0, T]$$
Discretize the equation according to $[D_t D_t u + \omega^2 u = f]^n$
    \subsection*{a) Derive the equation for the first time step}
        \begin{equation} \label{eq:1}
            {[D_t D_t u]}^n = \frac{u^{n+1} - 2u^n + u^{n-1}}{{\Delta t}^2}
        \end{equation}
        And then we get:
        \begin{equation}\label{eq:2}
            \frac{u^{n+1} - 2u^n + u^{n-1}}{{\Delta t}^2} + \omega^2 u = f(t)
        \end{equation}
        Solving equation (\ref{eq:2}) for $u^{n+1}$ we have:
        \begin{equation}
            u^{n+1} = (f(t_n) - \omega^2 u^n) \Delta t^2 + 2u^n - u^{n-1}
        \end{equation}
        Setting $n = 0$, to find $u^1$:
        \begin{equation}
            u^{1} = (f(t_0) - \omega^2 u^0) \Delta t^2 + 2u^0 - u^{-1}
        \end{equation}
        To find an expression to represent $u^{n-1}$ we can use $[D_{2t}u =  V]^0$, knowing that $u'(0) = V$:
        \begin{equation}
            \frac{u^{1} - u^{-1}}{2\Delta t} = V
        \end{equation}
        Solving the equation:
        \begin{equation}
            u^{-1} = u^{1} - 2V\Delta t
        \end{equation}
        Substituting $u^{-1} = u^{1} - 2V\Delta t$, and, $u(0) = I$ in equation (4), we get:
        \begin{equation}
            u^{1} = (f(t_0) - \omega^2 I) \Delta t^2 + 2I - u^{1} + 2V\Delta t
        \end{equation}
        Simplifying the last equation we get:
        \begin{equation}
            u^{1} = (f(t_0) - \omega^2 I) \frac{\Delta t^2}{2} + V\Delta t + I
        \end{equation}
    
    \newpage
    \subsection*{b) For verification purposes, we use the method of manufactured solutions (MMS) with the choice of $u_e(x,t)=ct+d$. Find restrictions on c and d from the initial conditions. Compute the corresponding source term f by term. Show that $[D_t D_t t]^n=0$ and use the fact that the $D_t D_t$ operator is linear, $[D_t D_t(ct + d)]^n = c[D_t D_t t]n+[D_t D_t d]n = 0$, to show that $u_e$ is also a perfect solution of the discrete equations.}
    
    Based on the initial conditions given $u(0) = I$, and $u'(0) = V$.
    For $t = 0$:
    \begin{equation}
        u_e(0) = c\times0 + d =  d = u(0) = I
    \end{equation}
    \begin{equation}\label{eq:10}
        \frac{d}{dt}u_e(t) = u_e(t) = c, u_e'(0) = e'(0) = V 
    \end{equation}
    Therefore we get:  $d = I$, and $c = V$
    To find an expression for $f$, we use the definition: $u" + \omega^2 u = f(t)$
    \newline
    Based on the result of the equation (\ref{eq:10})$\frac{d}{dt}u'_e(t) = \frac{d}{dt}c = 0$
    
    \begin{equation}
        u" + \omega^2 u = 0 + \omega^2 (ct + d) = \omega^2 (Vt + I) = f(t) 
    \end{equation}
    
    It us easy to show that $[D_t D_t t]^n = 0$, if we set $u = t$ in the equation (\ref{eq:1})
    
    \begin{equation}
            {[D_t D_t t]}^n = \frac{t^{n+1} - 2t^n + t^{n-1}}{{\Delta t}^2} = \frac{(t^{n+1} - t^n) - (t^n - t^{n-1})}{{\Delta t}^2} = \frac{\Delta t - \Delta t}{{\Delta t}^2} = 0
    \end{equation}
    
    Knowing that Operator $D_t D_t$ is linear, we can show that $u_e = Vt + I$ is also a perfect solution of the discrete equation, having $\omega^2 (Vt + I) = f$
    \begin{equation}
        [D_t D_t u + \omega^2u -f]^n = 0
    \end{equation}
    
    \begin{equation}
        [D_t D_t (Vt + I) + \omega^2 (Vt + I) - \omega^2 (Vt + I)]^n = [D_t D_t (Vt + I)]^n = V[D_t D_t t]^n + [D_t D_t I]^n = 0
    \end{equation}
    
    \newpage
    \subsection*{ d) The purpose now is to choose a quadratic function $u_e=b t^2 + ct + d$ as exact solution. Extend the sympy code above with a function quadratic for fitting f and checking if the discrete equations are fulfilled. (The function is very similar to linear.)}
    
    Running the program with the quadratic function would produce zero residual for both first step, and general equation. Therefore we can say that the quadratic solution is an exact solution.
    
    
    \subsection*{e) Will a polynomial of degree three fulfill the discrete equations?}
    
    Running the program with a polynomial of a degree three (poly d3), would produce 0 for the general form, but it is not zero for the first step ($-a*dt^3$). Which means that it won't fulfill the discrete equations.
    
    
    \subsection*{f) Write a nose test for checking that the quadratic solution is computed to correctly}
    
    The function nose-test in the python code, would print out the maximum absolute error, of the quadratic function. I wasn't sure how to initialize the variables, so I set them all to one. 
    
\end{document}
